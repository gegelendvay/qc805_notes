\documentclass[openany, a4, 12pt]{article}

\usepackage{amsfonts}
\usepackage{amsmath}
\usepackage{amsthm}
\usepackage{braket}
\usepackage{graphics}

%\setlength\parindent{0pt}

\title{Quantum Cryptography and Quantum Key Distribution Notes}
\author{Gergely Lendvay}
\date{\today}

\begin{document}
\maketitle\clearpage

\tableofcontents\clearpage

\section*{List of Acronyms}
\begin{tabbing}
\hspace{3cm} \= \kill
    AES \dotfill \> Advanced Encryption Standard \\
    OTP \dotfill \> One-Time Pad \\
    PQC \dotfill \> Post-Quantum Cryptography \\
    QKD \dotfill \> Quantum Key Distribution \\
    RSA \dotfill \> Rivest--Shamir--Adleman
\end{tabbing}

\newpage\section*{Introduction}
\subsection{Course Info}
\subsubsection{Teachers}
\noindent Konstantin Wernli -- kwernli@imada.sdu.dk

\noindent William Mistegard -- wem@imada.sdu.dk

\noindent Jurek Bleischwitz -- jurek@imada.sdu.dk

\subsubsection{Exercise Classes}
\begin{enumerate}
    \item Solutions must be uploaded before each class.
    \item Exercise sheets are discussed during the classes.
    \item The four best exercise solutions count towards the overall course grade.
\end{enumerate}

\subsubsection{Exam}
\begin{enumerate}
    \item Assessment is based on four graded exercise sheets, and
    \item Final oral exam consisting of:
    \begin{itemize}
        \item 10--12 minutes presentation,
        \item 13--15 minutes discussion and questions,
        \item 5 possible topics, with 30 minutes of preparation time.
    \end{itemize}
    \item Tentative exam date: June 9, 2026
\end{enumerate}

\subsubsection{Books}
Quantum Key Distribution -- Ramona Wolf

\subsection{Plan}
The course begins with an introduction to classical cryptography and quantum key distribution (QKD), followed by a recap of quantum computing and quantum information theory.
Subsequent topics include post-processing, security analysis, practical implementations of QKD, and post-quantum cryptography.

\newpage\section{First Lecture}
\subsection{Classical Cryptography}
The fundamental problem of cryptography is to enable secure communication between two parties, traditionally called Alice and Bob, while preventing an adversary Eve from gaining any information about the transmitted message.

Classical cryptography includes examples, such as the Caesar cipher, while more modern schemes can be categorized either by symmetric or asymmetric schemes. Well-known examples include Advanced Encryption Standard (AES) and Rivest--Shamir--Adleman (RSA).

\begin{table}[h!]
    \centering
    \resizebox{\textwidth}{!}{
        \begin{tabular}{|c|*{26}{c}|}
            \hline
            \textbf{Plain Alphabet} & A & B & C & D & E & F & G & H & I & J & K & L & M & N & O & P & Q & R & S & T & U & V & W & X & Y & Z \\
            \hline
            \textbf{Cipher Alphabet} & D & E & F & G & H & I & J & K & L & M & N & O & P & Q & R & S & T & U & V & W & X & Y & Z & A & B & C \\
            \hline
        \end{tabular}
    }
    \caption{Caesar Cipher Table (Right Shift by 3)}
    \label{table:caesar}
\end{table}

\paragraph{Impact of Quantum Computing on Classical Cryptography}
One of the major implications of quantum computing is its ability to efficiently solve certain mathematical problems that are believed to be hard for classical computers.
In particular, efficient prime factorization through Shor's algorithm would render widely used public-key cryptographic schemes such as RSA insecure.

\subsection{RSA Encryption Scheme}
The RSA algorithm was introduced in 1977 by Rivest, Shamir, and Adleman and is still widely used, making up approximately 50\% of all asymmetric encryption schemes.
The goal is to find 3 integers $n$, $e$, and $d$ such that for all messages
\[x\in \{0,\dots,n-1\},\]
the following holds:
\[(x^e)^d \equiv x\pmod{n}\]

\paragraph{Key Generation}
Two large random prime numbers $p$ and $q$ are chosen, typically satisfying
\[2^{1023} < p,q < 2^{1024}\]
The public modulus is defined as
\[n = pq\]
Let $\lambda(n)$ denote Carmichael's function,
\[\lambda(n) = \mathrm{km}(p-1,q-1)\]
An integer $e$ is chosen such that
\[1<e<\lambda(n), \qquad \gcd(e, \lambda(n)) = 1\]
The private key $d$ is defined as the modular inverse of $e$ modulo $\lambda(n)$:
\[d \equiv e^{-1} \pmod{\lambda(n)}\]

\paragraph{Encryption and Decryption}
The public key $(n,e)$ is made available to everyone, while the private key $d$ is kept secret.
To send a message $x$, Bob computes the ciphertext
\[y \equiv x^e \pmod{n}\]
and sends $y$ to Alice.
Using her private key, Alice decrypts the message by computing
\[y^d = (x^e)^d \equiv x \pmod{n}\]

An eavesdropper Eve, even knowing $n$, $e$, and the encrypted message $y$, cannot efficiently recover $x$ without access to $d$.
This is because computing $d$ requires knowledge of $\lambda(n)$, which in turn requires factoring $n$ into $p$ and $q$.

\paragraph{Proof}
Since $n = pq$,
\[m^{ed} \equiv m \pmod{p} \quad \text{and} \quad m^{ed} \equiv m \pmod{q}\]
Because
\[ed \equiv 1 \pmod{\lambda(n)},\]
there exists an integer $k$ such that
\[ed-1=k(p-1)\]
Then, for $x \not\equiv 0 \pmod{p}$,
\[m^{ed-1} = m^{k(p-1)} = (m^{p-1})^k = 1^k \equiv 1 \pmod{p},\]
by Fermat's Theorem, which states that if $p$ is a prime and $\gcd(m,p)=1$, then
\[m^{p-1} \equiv 1 \pmod{p}\]
The same argument holds for $q$.

\paragraph{Security Assumptions}
The security of RSA relies on the computational hardness of factoring large integers.
The best known classical factoring algorithms run in sub-exponential time,
\[\exp( c (\log n)^{1/3} (\log\log n)^{2/3}),\]
which is infeasible for sufficiently large $n$.

\paragraph{Quantum Threat}
Shor's algorithm shows that a quantum computer can factor integers in polynomial time,
\[O((\log n)^3)\]
This would completely break RSA if large-scale, fault-tolerant quantum computers become available.

Currently, practical limitations such as qubit count, noise, and error correction prevent Shor's algorithm from being applied.
In practice, breaking RSA would require millions of qubits.

Even if RSA remains secure in the near term, cryptographic systems must be developed that are resistant to quantum attacks.
This includes both new classical schemes (post-quantum cryptography) and quantum approaches such as Quantum Key Distribution (QKD).

\newpage\section{Second Lecture}
\subsection{Provably Secure Cryptography}
An encryption scheme is considered secure if its security can be mathematically proven.

Classical public-key schemes such as RSA are not proven secure against all efficient (polynomial-time) adversaries.
Their security relies on unproven computational hardness assumptions, such as the difficulty of factoring large numbers.

\paragraph{One-Time Pad}
A fundamental example of provably secure encryption is the One-Time Pad (Vernam cipher, 1926).

\noindent Alice and Bob share a secret key
\[S_A=S_B \in \{0,1\}^n,\]
unknown to Eve, and encrypt a message $m \in \{0,1\}^n$ using bitwise XOR:
\[c_i = m_i \oplus (S_A)_i\]

\noindent Example:
\[S_A=S_B=101, \qquad m=011, \qquad c=110\]
Bob decrypts the message using his key:
\[m'_i = c_i \oplus (S_B)_i = (m_i \oplus (S_A)_i) \oplus (S_B)_i = m_i \oplus \big((S_A)_i \oplus (S_B)_i\big)\]
If $S_A = S_B$, then
\[m'_i = m_i \oplus 0 = m_i\]

\noindent The One-Time Pad (OTP) satisfies
\[P[M=m \mid C=c] = P[M=m],\]
meaning that the ciphertext $c$ reveals no information about the message $m$.
This holds if the following conditions are satisfied:
\begin{enumerate}
    \item The key is uniformly random
    \item The key length equals the message length
    \item The key is never reused
    \item The key remains completely secret
\end{enumerate}

\noindent Ideally, the shared key satisfies:
\begin{enumerate}
    \item \textbf{Correctness:} $S_A=S_B$
    \item \textbf{Uniform Randomness:} $P[S=S] = 2^{-n}$
    \item \textbf{Secrecy:} Eve has no information about the keys $S_A$ and $S_B$
\end{enumerate}

\noindent In practice, small imperfections may occur:
\[P[S_A \neq S_B] < \varepsilon,\]
the key is statistically close to uniform, and the protocol aborts if eavesdropping is detected.

\subsection{Quantum Key Distribution}
Quantum Key Distribution enables two parties to generate a shared secret key with information-theoretic security, guaranteed by the laws of quantum mechanics.
The most well-known protocol is BB84, introduced by Bennett and Brassard in 1984.

The adversary Eve may listen to the classical channel and interact with the quantum channel, but cannot violate the principles of quantum mechanics.

\paragraph{Quantum Transmission Phase}
Alice selects two random bit strings
\[a,b \in \{0,1\}^n,\]
where $a$ is the raw key and $b$ determines the encoding basis.
Each bit $a_i$ is encoded into a qubit $\ket{\psi_i}$ as follows:
\begin{enumerate}
    \item Computational basis ($b_i = 0$):
    \[0 \mapsto \ket{0}, \qquad 1 \mapsto \ket{1}\]
    \item Diagonal/Hadamard basis ($b_i = 1$):
    \[0 \mapsto \ket{+} = \frac{1}{\sqrt{2}}(\ket{0}+\ket{1}), \qquad 1 \mapsto \ket{-} = \frac{1}{\sqrt{2}}(\ket{0}-\ket{1})\]
\end{enumerate}
Alice sends the qubits to Bob over the quantum channel.

\paragraph{Measurement Phase}
Bob independently chooses a random basis string
\[b' \in \{0,1\}^n,\]
and measures each received qubit in basis $b'_i$, obtaining outcomes $a'_i$.

\begin{itemize}
    \item If $b'_i = b_i (0 \to \ket{0})$, then $a'_i = a_i$ with probability 1.
    \item If $b'_i \neq b_i$, the measurement outcome is uniformly random and uncorrelated with $a_i$.
\end{itemize}

For example, if Alice sends $\ket{0}$ and Bob measures in the computational basis,
\[P_{\ket{0}}(0) = |\langle 0 \mid 0 \rangle|^2 = 1, \qquad P_{\ket{0}}(1) = 0,\]
and therefore Bob obtains the correct bit with probability $1$.

\noindent However, if Alice sends $\ket{+}$ but Bob measures in the computational basis,
\[P_{\ket{+}}(0) = |\langle 0 \mid + \rangle|^2 = \frac{1}{2}, \qquad P_{\ket{+}}(1) = \frac{1}{2},\]
Bob may only obtain the correct bit with probability $\frac{1}{2}$.

\paragraph{Sifting}
Alice and Bob publically compare the basis strings $b$ and $b'$ over the classical channel and discard all indices where $b_i \neq b'_i$.
The remaining bits form the sifted key, for which ideally (in the absence of noise and eavesdropping)
\[a_i = a'_i\]

\paragraph{Example}
The two tables below illustrate one run of the BB84 protocol.
Alice chooses random bits $a$ and random bases $b$ to prepare the quantum state $\psi$.
Bob independently chooses random measurement bases $b'$ and obtains outcomes $a'$.

\begin{table}[h!]
    \centering
    \begin{tabular}{|c|*{6}{c}|}
        \hline
        \textbf{a} & 0 & 1 & 1 & 1 & 0 & 1 \\
        \hline
        \textbf{b} & 1 & 0 & 1 & 1 & 0 & 1 \\
        \hline
        $\boldsymbol{\psi}$ & $\ket{+}$ & $\ket{1}$ & $\ket{-}$ & $\ket{-}$ & $\ket{0}$ & $\ket{-}$ \\
        \hline
    \end{tabular}
    \caption{Alice's side}
\end{table}

\clearpage
\begin{table}[h!]
    \centering
    \begin{tabular}{|c|*{6}{c}|}
        \hline
        \textbf{b'} & 1 & 0 & 0 & 0 & 1 & 1 \\
        \hline
        \textbf{a'} & 0 & 1 & 1 & 0 & 1 & 1 \\
        \hline
        \textbf{Basis Match} & Yes & Yes & No & No & No & Yes \\
        \hline
        \textbf{Sifted Key} & 0 & 1 & - & - & - & 1 \\
        \hline
    \end{tabular}
    \caption{Bob's side}
\end{table}

\noindent In this example, the sifted key is 011.

\paragraph{Security Principle}
The security of QKD comes from fundamental quantum properties:
\begin{itemize}
    \item \textbf{Measurement Disturbs the State:} Any attempt by Eve to measure the quantum states introduces detectable errors.
    \item \textbf{No-Cloning Theorem:} Unknown quantum states cannot be copied, preventing Eve from perfectly duplicating transmitted qubits.
\end{itemize}

\newpage\section{Third Lecture}

\end{document}
